\documentclass[10pt]{article}


\usepackage{times}
\usepackage{amsfonts}
\usepackage{amsmath}
\usepackage[psamsfonts]{amssymb}
\usepackage{latexsym}
\usepackage{color}
\usepackage{graphics}
\usepackage{enumerate}
\usepackage{amstext}
\usepackage{blkarray}
\usepackage{url}
\usepackage{epsfig}
\usepackage{bm}
\usepackage{hyperref}
\hypersetup{
    colorlinks=true,
    linkcolor=blue,
    filecolor=magenta,      
    urlcolor=blue,
}
\usepackage{mathtools}
 
\def\Kset{\mathbb{K}}
\def\Nset{\mathbb{N}}
\def\Qset{\mathbb{Q}}
\def\Rset{\mathbb{R}}
\def\Sset{\mathbb{S}}
\def\Zset{\mathbb{Z}}
\def\squareforqed{\hbox{\rlap{$\sqcap$}$\sqcup$}}
\def\qed{\ifmmode\squareforqed\else{\unskip\nobreak\hfil
\penalty50\hskip1em\null\nobreak\hfil\squareforqed
\parfillskip=0pt\finalhyphendemerits=0\endgraf}\fi}

\DeclareMathOperator*{\E}{\rm E}
\DeclareMathOperator*{\argmax}{\rm argmax}
\DeclareMathOperator*{\argmin}{\rm argmin}
\DeclareMathOperator{\sgn}{sign}
\DeclareMathOperator{\supp}{supp}
\DeclareMathOperator{\last}{last}
\DeclareMathOperator{\sign}{\sgn}
\DeclareMathOperator{\diag}{diag}
\providecommand{\abs}[1]{\lvert#1\rvert}
\providecommand{\norm}[1]{\lVert#1\rVert}
\def\vcdim{\textnormal{VCdim}}
\DeclareMathOperator*{\B}{\textbf{B}}
\DeclarePairedDelimiter\ceil{\lceil}{\rceil}
\DeclarePairedDelimiter\floor{\lfloor}{\rfloor}

\newcommand{\cX}{{\mathcal X}}
\newcommand{\cY}{{\mathcal Y}}
\newcommand{\cA}{{\mathcal A}}
\newcommand{\ignore}[1]{}
\newcommand{\bi}{\begin{itemize}}
\newcommand{\ei}{\end{itemize}}
\newcommand{\be}{\begin{enumerate}}
\newcommand{\ee}{\end{enumerate}}
\newcommand{\bd}{\begin{description}}
\newcommand{\ed}{\end{description}}
\newcommand{\h}{\widehat}
\newcommand{\e}{\epsilon}
\newcommand{\mat}[1]{{\mathbf #1}}
\newcommand{\R}{\mat{R}}
\newcommand{\0}{\mat{0}}
\newcommand{\M}{\mat{M}}

\newcommand{\SP}{\mathbf{S}_{+}^n}

\newcommand{\D}{\mat{D}}
\renewcommand{\r}{\mat{r}}
\newcommand{\x}{\mat{x}}
\renewcommand{\u}{\mat{u}}
\renewcommand{\v}{\mat{v}}
\newcommand{\w}{\mat{w}}
\renewcommand{\H}{\text{0}}
\newcommand{\T}{\text{1}}
\newcommand{\set}[1]{\{#1\}}
\newcommand{\xxi}{{\boldsymbol \xi}}
\newcommand{\ssigma}{{\boldsymbol \sigma}}
\newcommand{\Alpha}{{\boldsymbol \alpha}}
\newcommand{\tts}{\tt \small}
\newcommand{\hint}{\emph{hint}}
\newcommand{\matr}[1]{\bm{#1}}     % ISO complying version
\newcommand{\vect}[1]{\bm{#1}}     % ISO complying version

\newcommand{\Var}{\mathrm{Var}}
\newcommand{\Cov}{\mathrm{Cov}}


%\renewcommand{\labelitemi}{$\bullet$}
%\renewcommand{\labelitemii}{$\cdot$}
%\renewcommand{\labelitemiii}{$\diamond$}
%\renewcommand{\labelitemiv}{$\ast$}

\newenvironment{solution}{\vspace{.25cm}\noindent{\it Solution:}}{}

\begin{document}

\noindent MATH-GA.2012.001 Selected Topics in Numerical Analysis :\\
Convex and Nonsmooth Optimization, Spring 2020\\
Homework Assignment 1 \\
Yves Greatti - yg390\\

\begin{enumerate}
\item Prove that the quadratic cone is convex.
Given the quadratic cone C = \{(x,t) $\in \mathbf{R}^{n+1} | ||x||_2 \le t$ \}. By triangle inequality, and homogeneity for any $x1,x2 \in C$ and $\theta \in [0,1]$:
\begin{align*}
|| \theta 
		\begin{bmatrix}
    			x1 \\ 
			t
		\end{bmatrix} 
+ (1 - \theta) 
		\begin{bmatrix}
    			x2 \\ 
			t
		\end{bmatrix} ||_2 &\le 
|| \theta 
		\begin{bmatrix}
    			x1 \\ 
			t
		\end{bmatrix} ||_2 +	
|| (1 - \theta) 
		\begin{bmatrix}
    			x2 \\ 
			t
		\end{bmatrix} ||_2 \\
		&= \theta ||
		\begin{bmatrix}
    			x1 \\ 
			t
		\end{bmatrix} ||_2 +	
(1 - \theta) ||
		\begin{bmatrix}
    			x2 \\ 
			t
		\end{bmatrix} ||_2 \\
		& \le \theta t + (1-\theta) t \\
		&= t	
\end{align*}

\item Prove (using the definition of convexity) that the intersection of two convex sets is convex. (See BV p.36) 
Let C1, C2 two convex sets and $C3 = C1 \cap C2$. For any $x1, x2 \in C3$ and $\theta \in [0,1]$:
\begin{align*}
	\theta x1 + (1-\theta) x2 \in C1 \text{ since } x1, x2  \in C1 \\
	\theta x1 + (1-\theta) x2 \in C2 \text{ since } x1, x2  \in C2 \\
	\Rightarrow \theta x1 + (1-\theta) x2 \in C1 \cap C2 = C3
\end{align*}
C3 is convex.

\item Prove that the image of a convex set under an affine function is convex, and that the inverse image is also convex.
Given f an affine function, $f: \mathbf{R}^n \rightarrow \mathbf{R}^m$,
with $f(x)=Ax + b$, where $A  \in \mathbf{R}^{m \times n}$, $b \in \mathbf{R}^m$ and suppose that $C \subseteq  \mathbf{R}^{n}$ is convex,

$\forall y_1, y_2 \in f(C), \forall \theta \in [0, 1]$, and let $f(x_1)=y_1, f(x_2) = y_2$, we have: 
\begin{align*}
	\theta y_1  + (1-\theta) y_2 	&= \theta f(x_1)  + (1-\theta) f(x_2) \\
							&= \theta (A x_1 + b) + (1-\theta) (A x_2 + b) \\
							&= \theta  A (x_1 + x_2)
\end{align*}
which is a linear combination of $x_1, x_2$ with b =0, and since C is convex $x_1, x_2 \in C$, so $\theta y_1  + (1-\theta) y_2 \in f(C)$ and $f(C)$ is convex.

Suppose now $\forall x_1, x_2 \in f^{-1}(C), \forall \theta \in [0, 1]$, and let $f(x_1)=y_1, f(x_2) = y_2$, with $y_1, y_2 \in C$, C is a convex set, we have: 
\begin{align*}
	f(\theta x_1  + (1-\theta) x_2) 	&= A [\theta x_1)  + (1-\theta) x_2]\\
							&= \theta (A x_1 + b) + (1-\theta) (A x_2 + b) \\
							&= \theta  y_1 + (1-\theta) y_2
\end{align*}
Since C is convex then $y_3 =\theta  y_1 + (1-\theta) y_2$ is also in C. Therefore we showed that there exist $y_3 \in C$ such that $f(\theta x_1  + (1-\theta) x_2) = y_3$
which proves that $f^{-1}(C)$ is convex.

\item BV Ex 2.1
Let $C \subseteq  \mathbf{R}^{n}$ be a convex set, $x1_, \cdots, x_k \in C$ and  $\theta1, \cdots, \theta_k \in C$, with $\theta_i \ge 0$ and $\sum_i \theta_i = 1$.
Then by definition of the convexity, for k=2, $\sum_{i=1}^k \theta_i x_i \in C$ holds. Assuming this is also true for k=n-1, then $\sum_{i=1}^n \theta_i x_i = (\sum_{i=1}^{n-1} \theta_i x_i) + \theta_n x_n$, 
which is the sum of two elements of C which is in C by induction.

\item BV Ex 2.2
Show that a set is convex if and only if its intersection with any line is convex. Show that a set is affine if and only if its intersection with any line is affine.

If C is a convex set, a line being affine is also convex and the intersection will be convex. 
If the intersection of a set with a line is convex and non empty, any points of C will also be in the intersection therefore in C.
The same applies to affine set since any affine set is convex.

\item BV Ex 2.10
Let $C \subseteq  \mathbf{R}^{n}$, the solution set of a quadratic inequality,
$$ C =\{x \in  \mathbf{R}^{n} | x^T A x + b^T x + c \le 0 \}$$
with $A \in  \mathbf{S}^{n}$,$b \in  \mathbf{R}^{n}$, and $c \in  \mathbf{R}$.
\be
	\item Show that C is convex if $A \succeq 0$
	\item  Show that the intersection of C and the hyperplane defined by $g^T x + h = 0$ (where $g \neq 0$) is convex if $A + \lambda g g^T  \succeq 0$ for some $\lambda \in \mathbf{R}$.
\ee
Are  the converses of these statements true?


source: math.stackexchange.com

\be
	\item 
	rewriting $C$ as $C =\{x \in  \mathbf{R}^{n} | (x^T A x) + (b^T x) \le \alpha, \alpha \in   \mathbf{R} \}$.
	Then the condition on x is the sum of two convex functions $x^T A x$ if $A \succeq 0$ and $b^T x$
	and sublevels set of a function are convex (BV 3.1.6).
	If $A=-1, b=0, c=-1$, $C=\{x \in  \mathbf{R}^{n} | \|x\|_2^2 \ge 1 \}$ is convex but A is not positive semi-definite so the converse is not true.
	
	\item First we show that the intersection of C with a line is convex when $A \succeq 0$.
	
	Let $l= \{x + tv | t \in \mathbf{R} \}$ an arbitrary line, replacing any point of this line in $C \cap l$, we have:

	$$(x + tv)^T A (x + tv) + b^T (x + tv) + c = (v^T A v) t^2 + (2 x^T A v + b^T v) t + x^T A x + b^T x + c$$
	
	$C \cap l = \{ x | \alpha t^2 + \beta t + \gamma \le 0 \text{ where } \alpha = v^T A v, \beta = 2 x^T A v + b^T v \text{ and } \gamma = x^T A x + b^T x + c \}$.
	It is the equation of a parabola, which opens upward towards $+\infty$ when $\alpha > 0$ and the points solution are all the points for which the quadratic equation is negative or zero; it is a bounded interval and convex. 
	When $\alpha = 0$ the equation is $\beta t + \gamma$ which is affine and convex. And when 
	$\alpha < 0$ the parabola is open downward towards $-\infty$ and the solutions are the union of two disjoint intervals and is not convex.
	Thus $C \cap l$ is convex when $\alpha =  v^T A v \ge 0$ thus C is convex when $A \succeq 0$.
	WLOG we now consider $C \cap l \cap H$, and notice:
	\begin{align*}
		g^T\; . \; (x + t \; v) + h	&=	0 \\
		g^T \; v \; t				&= 0	\text{ since } g^T \; x + h = 0
	\end{align*}
	So we are looking for points in $I = C \cap l \cap H = \{ x |  \alpha t^2 + \beta t + \gamma \le 0, \epsilon t = 0, \text{ with the same as above } \alpha, \beta, \gamma, \epsilon=g^T \; v \}$.
	If $t = 0$ then the intersection reduces to the point $\{ x \}$ assuming $\gamma \le 0$ or the empty set, in both cases the intersection is convex. if $g^T v = 0$ then the intersection is now 
	$I =\{ x |  \alpha t^2 + \beta t + \gamma \le 0 \}$ and this is verified when $A \succeq 0$. Since $g g^T  \succeq 0$, we conclude that $I$ is convex if
	$A  \succeq 0  \Rightarrow (A + \lambda g g^T) \succeq 0, \lambda \ge 0$. The converse is not verified for the same counter-example as (a).
	

\ee



\item BV Ex 2.16

Let $S_1$, $S_2$ two convex sets $\in \mathbf{R}^{m + n}$ and $S=\{ (x, y_1 + y_2) | x \in \mathbf{R}^{m}, y_1, y_2 \in \mathbf{R}^{n}, (x, y_1) \in S_1, (x, y_2) \in S_2 \}$.
$\forall (x,y_1 + y_2) \in S, (x, z_1 + z_2) \in S, \forall \theta \in [0, 1]$, we have: $\theta (x, y_1 + y_2) + (1-\theta) (x, z_1 + z_2)  = (x, (\theta y_1 + (1-\theta) y_2) +  (\theta z_1 + (1-\theta) z_2))$ which is in the form
$(x, t + s)$ where $x \in \mathbf{R}^{m}, t = (\theta y_1 + (1-\theta) y_2) \in S_1, s = (\theta z_1 + (1-\theta) z_2) \in S_2$ since $S_1$, $S_2$ are convex. Thus S is convex.

\item BV Ex 2.23 Give an example of two closed convex sets that are disjoint but cannot be strictly separated.
$S_1 = \{ x \in \mathbf{R}^{2}: x_1 > 0, x_2 \ge \frac{1}{x_1}\}$ and $S_2 = \{ x \in \mathbf{R}^{2}: x_2 = 0 \}$.
 $S_1$ and $S_2$ are closed, convex, and disjoints. 
 Any line of separating the two sets must be of the form $[0 1]^T x = \beta$ but $ [0 1]^T b = 0 $ for all $b\in S_2$, on the
 other hand $\inf_{a \in S_1} [0 1]^T a = 0$, this implies there cannot be strict separation.
 
%$\lim_{x\to0^+} \frac{1}{x} = +\infty$ thus it is impossible to separate the two sets $\{ (y | y = \frac{1}{x} , x > 0 \}$ and the y-axis.

\item BV Ex 2.24 (b)  Supporting hyperplanes.\\
Let $C=\{x \in \mathbf{R}^{n} | \| x \|_{\infty} \le 1\}$ and let $\hat{x}$ be a point in the boundary of C. Identify the supporting hyperplanes of C at $\hat{x}$ explicitly.

By definition if C is supported at $\hat{x}$ iff $\exists v \in \mathbf{R}^{n}, v \neq 0$ such that $v^T. a \ge v ^T. \hat{x}$ for all $a \in C$. 
If $ \|\hat{x}\| = 1$, and $\hat{x} =1$ then we take v = -1, 
if $ \|\hat{x}\| = 1$, and $\hat{x} =-1$ then we take v = 1, and $ \|\hat{x}\| \le 1$, with  $-1 < \hat{x}  <1$ then we take v = 0.

source:  https://pages.wustl.edu/files/pages/imce/nachbar/convexityrn.pdf

\item Verify that as stated on BV p.39, the hyperbolic cone is the inverse image of the second order cone under the given affine transformation. 
Let $C$, the hyperbolic cone: $C=\{x | x^T P x \le (c^T x)^2; c^t x \ge 0 \}$ where $P \in \SP$ and $c \in \mathbf{R}^{n}$, and 
$S$, the second-order cone: $S=\{ (z,t) | z^T z \le t^2; t \ge 0 \}$.
For any point $x$ of $C$, we want to show that under affine function $f(x) = (P^{\frac{1}{2}} x, c^T x)$, $C = \{x | f(x) \in S \}$.
$(P^{\frac{1}{2}} x)^T (P^{\frac{1}{2}} x) = x^T (P^{\frac{1}{2}})^T P^{\frac{1}{2}} x = x^T P^{\frac{1}{2}} P^{\frac{1}{2}} x$ since $P$ is symmetric.
So $C = \{x | \| P^{\frac{1}{2}} x \|_2^2 \le (c^T x)^2 \}$ or $C = \{(x, ct) | \| P^{\frac{1}{2}} x \|_2^2 \le (c^T x)^2,  (c^T x) \ge 0\}$, $C = \{(x, ct) | f(x) \in S \}$.
\end{enumerate}

\end{document}
